\section{Einleitung}
\subsection{Themenvorstellung}
Immer mehr Unternehmen steigen in ihrer internen Abwicklung von Projekten auf sogenannte agile Prozessmodelle um. Das in dieser Seminararbeit behandelte Unternehmen steht nun vor der Frage, ob dessen Betriebsprozesse auf eines dieser Modelle umgestellt werden sollen. Aus einem Interview mit dem Bereichsleiter für Softwareentwicklung B. R. geht hervor, dass bereits interne Gespräche stattfanden und einige Anforderungen getroffen wurden. Da die Prozesse des Unternehmens bereits veraltet sind und eindeutig Verbesserungspotential besteht, wird die Prüfung, ob eines der agilen Prozessmodelle die Anforderungen der Geschäftsleitung erfüllen kann, zur Motivation für diese Arbeit genutzt.TODO Zitat

\subsection{Zielsetzung}
Mit dieser Seminararbeit wird das Ziel verfolgt, die Prozesse des gewählten Unternehmens zu optimieren bzw. an die gestellten Anforderungen anzupassen. Die veralteten und suboptimalen Vorgehensweisen in der Projektabwicklung sollen aktuellen, effizienteren Prozessen weichen, um die Vorteile der Agilität im Management von Projekten maximal ausnutzen zu können. Vorstellbar wären hier unter anderem Verbesserungen im Bereich der Flexibilität, Ressourcen- und Zeitausnutzung bzw. Wirtschaftlichkeit und Kostensenkung, um nur einige zu nennen. Dem Unternehmen soll letztendlich durch Evaluation ausgewählter agiler Prozessmodelle das für ihre Bedürfnisse und Vorstellungen auf Basis der definierten Anforderungen bestmögliche Modell geliefert werden.

\subsection{Aufbau}
Zu Beginn wird die aktuelle Prozessstruktur aufgezeigt. Es werden alle Schritte, die von der Auftragsakquisition bis hin zur Inbetriebnahme des fertigen Softwareproduktes durchlaufen werden, analysiert. Anschließend werden anhand eines Interviews mit dem zuständigen Bereichsleiter des Unternehmens die gewünschten Anforderungen spezifiziert und priorisiert. Außerdem werden einige grundlegende Punkte zum Thema Agilität und agiler Softwareentwicklung erklärt und drei weit verbreitet agile Prozessmodelle vorgestellt, und zwar XP (Extreme Programming), FDD (Feature Driven Development) und Scrum. Diese werden zunächst grob erläutert, deren wichtigste Bestandteile und Merkmale dargelegt und daraufhin der Ablauf, welcher im Zuge eines Projekts je Prozessmodell verfolgt wird, aufgezeigt. Um ein für das Unternehmen geeignetes Modell bestimmen zu können, werden die zuvor vorgestellten im Rahmen der definierten Anforderungen evaluiert. Sollte keines der genannten Modelle die Anforderungen hinreichend erfüllen, wird versucht, aus den gegebenen ein passendes hybrides Prozessmodell zusammenzuführen. Abschließend werden die gewonnen Erkenntnisse noch einmal reflektiert und es wird ein Ausblick darauf gewährt, inwieweit sich das gewählte Modell in Zukunft optimieren lässt und welche weiteren Anforderungen gestellt werden könnten.