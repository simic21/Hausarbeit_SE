\section*{Interview vom 01.07.2017}
\textit{Mit welcher Art von Projekten befasst sich Ihr Unternehmen?}

Wir entwickeln in unserem Unternehmen unser eigenes Softwareprodukt, welches wir im Standard verkaufen oder auf Wunsch von Kunden nach deren Vorstellungen personalisieren oder erweitern.

\textit{Warum sollen die Betriebsprozesse umgestellt werden?}

Man muss ja mit der Zeit gehen. Viele bekannte Unternehmen, unter anderem auch einige unserer Partner, stellen Stück für Stück auf agile Prozesse um. Wir arbeiten nun schon seit knapp einem Jahrzehnt nach den gleichen Abläufen, ohne diese je verbessert zu haben. Sowohl die Geschäftsführung als auch die Mitarbeiter sind sich einig, dass diese nicht nur veraltet sind, sondern auch bezüglich der Effizienz des Ressourcenverbrauchs und Zeitausnutzung deutliche Defizite vorweisen. Nach gemeinsam Gesprächen haben wir uns schließlich zu diesem Schritt entschieden.

\textit{Welche Anforderungen sollen diese agilen Prozesse zukünftig abdecken können?}

Zunächst stehen natürlich die Gründe, die uns zur Umstellung bewegen, im Vordergrund. Sprich eine optimierte Ressourcen- und Zeitausnutzung ist hier definitiv die Hauptanforderung. Unsere Mitarbeiter sind, besonders in verhältnismäßig großen Projekten, oft zu stark ausgelastet und kommen mit der Arbeit nicht hinterher. Da wir immer mindestens drei-vier Aufträge gleichzeitig bearbeiten, sind wir gezwungen, den Mitarbeitern mehrere Aufgaben verschiedener Projekte zuzuweisen. Das führt natürlich unweigerlich dazu, dass ab und an Aufgaben zu spät fertiggestellt werden oder sogar komplett liegen bleiben. Dies wollen wir natürlich verhindern und hoffen, dass durch eine gezieltere Aufgabenteilung und Teambildung die Mitarbeiter konzentrierter an einer bestimmten Aufgabe bzw. Projekt arbeiten können und sich dadurch die Produktivität in der Entwicklung steigert und die Entwicklungsdauer im Gesamten verkürzt wird.

Da wir zurzeit insgesamt nur ca. 35 Entwickler zur Verfügung haben, sollten die Prozesse natürlich eher auf kleine Projektteams abgestimmt sein, wenn wir laufend mehrere Aufträge bearbeiten und somit mehrere abgegrenzte Teams bilden müssen. Einige dieser Mitarbeiter sind noch in der Ausbildung oder vor Kurzem erst dazugestoßen und müssen alle von uns entwickelten Programme kennen lernen. Ebenso soll das Wissen maximal unter den Mitarbeitern verteilt werden, da im Krankheitsfall einer Person jederzeit ein anderer Entwickler seine Tätigkeiten übernehmen können muss. Im Moment existieren für manche Komponenten unseres Produktes nur einzelne Spezialisten. Dies muss definitiv geändert werden. Wünschenswert wäre es, wenn jeder Entwickler das Wissen besitzen würde, jede Komponente bearbeiten zu können.

Natürlich bestehen nicht nur Engpässe in der Entwicklung. Es kann ebenso vorkommen, dass während der Projektplanung oder der Qualitätssicherung Zeitverzögerungen auftreten. Daher ist es wünschenswert, dass durch die Optimierung der Prozessabläufe durch ein einheitliches Vorgehen die gesamte Durchlaufzeit verkürzt wird.
Es kommt außerdem vermehrt vor, dass sich Projektspezifikationen, die in Absprache mit dem Kunden erstellt werden, während der Bearbeitung als nicht realisierbar darstellen oder nicht detailliert genug beschrieben werden. Da zurzeit lediglich zu Beginn des Projektes eine Abstimmung mit dem Kunden stattfindet, wird dadurch ein erheblicher Mehraufwand erzeugt, wenn der betroffene Entwickler den Kunden erneut selbst oder sogar über den Umweg des Teamleiters kontaktieren muss. Daher soll zukünftig möglichst ein stetiger Austausch zwischen Kunden und Projektteam stattfinden können.

Auch wenn agile Softwareentwicklung durch geringen Dokumentationsaufwand geprägt ist, wird dennoch zumindest eine Anwenderdokumentation benötigt. Es muss festgehalten werden, wie das Programm zu bedienen ist und wie genau die Kundenanpassung umgesetzt wurde. Dies unterstützt im Falle weiterer gewünschter Anpassungen die Entwicklung und verringert den Einarbeitungsaufwand.

Weiterhin würden wir uns gerne durch ein integriertes Risikomanagement vor möglichen Risiken absichern können, da aktuell weder eine Prüfung auftretender Risiken noch die Aufstellung entsprechender Gegenmaßnahmen durchgeführt wird. Deswegen verfallen wir immer öfter nach z.B. plötzlichen Mitarbeiterausfällen oder verspäteten Hardwarelieferungen in völlige Aufruhr und suchen lange nach Lösungsmaßnahmen. Mit einem Risikomanagement ließe sich dies durch Notfallpläne vermindern oder sogar ganz vermeiden.

\textit{Haben Sie bereits an eine bestimmte Lösung gedacht?}

Nicht zwangsläufig. Natürlich sind mir die gängigen agilen Prozessmodell wie Scrum oder Kanban ein Begriff, die genauen Vorgehensweisen sowie deren Vor- und Nachteile kenne ich allerdings nicht. Ob eines dieser Modelle, ein völlig anderes und unbekanntes Modell oder sogar eine Kombination verschiedener Modelle den Anforderungen entspricht muss erst noch geprüft werden.
