\newpage
\section{Optimierung des Softwareentwicklungsprozesses}
\subsection{Evaluation der Modelle}
Da jedes der vorgestellten Modelle zu den agilen Prozessmodellen zählt, ist der Projektablauf im Grundgerüst ähnlich. Die Software wird in iterativen Durchläufen entwickelt, die dadurch entstehenden Inkremente werden an den Kunden ausgeliefert und das dadurch erhaltene Feedback wird in irgendeiner Form in den Entwicklungsprozess zur Optimierung einbezogen. Die Anforderungen werden möglichst klein gehalten, gesammelt, geschätzt und einzelnen Entwicklern zugwiesen. Alle Modelle sind dabei auf kleine Teams ausgerichtet.

In XP ist das Hauptziel die Entwicklung guter und lauffähiger Software. Dabei wird die Dokumentation außer Acht gelassen, das Wissen wird lediglich über die Kommunikation und Rotation der Teams weitergereicht.\footcite[Vgl.][Seite 168]{cockburn} Die einzige Form der Dokumentation erfolgt über die Sammlung der Anforderungen in möglichst kleinen User Stories. Zur Klärung der aufkommenden Fragen wird ein Kunde vor Ort einberufen und schränkt somit den Kommunikationsaufwand ein, sofern es möglich ist, einen kundenseitigen Ansprechpartner abstellen zu können. Die Entwicklung erfolgt immer paarweise pro Anforderung und Arbeitsplatz, wodurch das Wissen zwar breiter gestreut wird, dadurch allerdings auch doppelte Kosten für den gleichen Aufwand entstehen. Im Gegenzug wird die Qualität des Quellcodes durch zwei Entwickler besser gewährleistet.

Beim FDD wird zu Beginn des Projekts ein allgemeines Modell des Zielsystems aufgestellt, wodurch bereits eine grobe Dokumentation geliefert wird. Aus diesem heraus werden dann die einzelnen Anforderungen definiert. Mit dem Chief Architect, Chief Programmer und Domänenexperten und -teams werden für die Mitarbeiter bereits ihre Aufgaben und Rollen definiert. Dazu werden die Anforderungen nicht nur in ihrer kleinsten Ausführung dargestellt, sondern werden in mehrere Ebenen aufgeteilt und zusammengefasst, wodurch die Übersichtlichkeit über das Zielsystem erhöht wird. Da mit dem Kunden nur auf der Geschäftsaktivitätenebene kommuniziert wird, können Anforderungen zwar gebündelt werden, es fehlt aber das Feedback zu einzelnen Features. Die Bündelung in Arbeitspakete und die Definition der Klassen und Modelle liefern zusätzlich weitere Dokumentationen.

Scrum definiert die Aufgaben und Rollen der Mitarbeiter klar, Verantwortlichkeiten werden eindeutig verteilt. Die Vielzahl an Meetings senkt zwar die effektive Arbeitszeit, erhöht allerdings die Qualität der Entwicklungsprozesse. Durch die Erstellung der Anwenderdokumentation und die laufende Pflege des Product Backlogs wird trotz agiler Softwareentwicklung die maximale Dokumentationsmenge geliefert. Die Eigenständigkeit und Zusammenarbeit des Teams erhöht das Verantwortungsbewusstsein und streut das Wissen innerhalb des Teams. Mit der Aufteilung der Iterationen in Sprints, welche ihre eigenen Meetings erhalten, sind die Entwicklungsschritte klar definiert und intensiv kommuniziert.

\subsection{Auswahl einer Lösung}
Die Hauptanforderung des Unternehmens, die optimierte Ressourcen- und Zeitausnutzung, wird von allen Modellen aufgrund ihrer Definition als agile Prozessmodelle gleichermaßen erfüllt. Ebenso werden die Aufgaben klar verteilt, allerdings in unterschiedlicher Ausprägung. In FDD und Scrum existieren Rollen für die einzelnen Mitarbeiter, in XP gibt es so eine Verteilung nicht. Im Gegenzug ist die Teambildung durch das Zusammensitzen der Mitglieder eines Teams im XP am eindeutigsten, bei Scrum werden die Teams ebenfalls klar definiert. Jedes der Modelle favorisiert außerdem eher kleine Teams.

Durch die Bearbeitung der Anforderungen in Paaren, ist die Wissensstreuung bei XP am größten. Da allerdings nur sehr wenig Programmierer verfügbar sind, ist diese Art der Entwicklung nicht umsetzbar. Scrum streut das Wissen durch die autarke und enge Zusammenarbeit der Teams ebenfalls sehr gut. Die Minimierung des Kommunikationsaufwandes wird am effektivsten durch die Abstellung eines Kunden vor Ort erreicht, wie es bei XP praktiziert wird. Dokumentationen werden allerdings lediglich bei FDD und Scrum erstellt, wobei Scrum hier ausführlicher dokumentiert.

XP entfällt als mögliche Lösung aufgrund der fehlenden Umsetzbarkeit von Pair Programming und der nicht vorhandenen Dokumentation. FDD erfüllt zwar einen Großteil der Anforderungen, allerdings nicht in der gewünschten Tiefe. Das Wissen wird nicht optimal unter den einzelnen Mitarbeitern gestreut und es existiert keine klare Teambildung. Da Scrum alle Anforderungen abdeckt, wird diese Modell zur Umsetzung gewählt. Einige Aspekte anderer Modelle lösen einige Anforderungen aber besser, daher werden diese in Scrum integriert. Zum einen ist das Pair Programming von XP eine gute Lösung, Wissen an Auszubildende oder neue Mitarbeiter weiterzugeben. Hierbei werden lediglich diese Mitarbeiter bestimmten Spezialisten zugewiesen, um ihnen über die Schulter schauen zu können und so einen größeren Lernerfolg zu erhalten. Zum anderen sollte ein kundenseitiger Ansprechpartner zur Senkung des Kommunikationsaufwandes immer vor Ort sein, wenn die Möglichkeit dazu besteht.