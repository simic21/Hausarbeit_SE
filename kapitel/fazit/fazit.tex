\newpage
\section{Schlussbetrachtung}
\subsection{Fazit}
Das vorgestellte Unternehmen ist ein mittelständisches Softwarehaus, welches ihr eigenes Softwareprodukt entwickelt und vermarktet. Es wird sowohl als Standardversion vertrieben als auch auf Kundenwunsch angepasst. Die dadurch entstehenden Projekte sind bezüglich der Durchführung nicht optimal gestaltet. Es fehlt an klarer Teambildung, Zuständigkeitsverteilung, Kommunikation und effektiver Ressourcen- und Zeitausnutzung. Durch die Einführung eines agilen Prozessmodells sollen diese und weitere Punkte optimiert werden.

Agile Softwareentwicklung wird geprägt von der Verwendung agiler, leichtgewichtiger Prozesse, welche die Flexibilität und Reaktionsgeschwindigkeit eines Unternehmens auf Anforderungsveränderungen fördern. Es existieren die unterschiedlichsten Prozessmodelle, welche diese Art der Entwicklung implementieren, sie alle verbinden aber die gleichen Grundaspekte wie iterative Entwicklungsphasen, die Konzentration auf qualitativ hochwertigen Programmcode und die Einfachheit der Lösungen.

XP konzentriert sein Augenmerk auf die Einfachheit des Codes, die Kommunikation zum Kunden und die paarweise Entwicklung des Produkts. FDD legt dagegen Wert auf die Modellierung. Es wird ein initiales Modell des Systems aufgestellt und nach der Aufgabenverteilung werden die zu implementierenden Klassen und Methoden modelliert. Scrum definiert klare Rollen für die Mitarbeiter und pflegt einen für ein agiles Prozessmodell ungewöhnlich großen Dokumentationsumfang. Nach einer Evaluation der einzelnen Modelle und der Prüfung, welches Modell den Anforderungsumfang am besten erfüllt, ist die Wahl auf Scrum mit der Integration einiger XP Elemente gefallen.

\subsection{Ausblick}
Für die entfernte Zukunft besteht das Ziel, zusätzlich zu den Anforderungen auch die Wünsche umzusetzen. Das gesamte Wissen über die Produktpalette des Unternehmens an jeden Entwickler zu übertragen, wird durch den zukünftigen Einsatz von Scrum automatisch geschehen. Für die Umsetzung des Risikomanagements könnte die Einführung einer weiteren Rolle zu den drei klassischen Rollen Product Owner, ScrumMaster und dem Team in Erwägung gezogen werden, welche für die Risikomanagement-Tätigkeiten verantwortlich wäre.\footcite[Vgl.][Seite 197]{nelson}